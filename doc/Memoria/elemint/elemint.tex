En \LaTeXe se denomina entorno flotante a aquel en el que es el compilador el que decide el lugar más conveniente para situarlos. Por eso es importante que todos estos elementos tengan una etiqueta (\textsl{label}) y que en el texto sean referenciados. Nunca hay que utilizar las expresiones `La siguiente figura ...' o `La siguiente tabla ...' sino que hay que utilizar expresiones como `En la figura 1.3' o `En la tabla 2.7'. Para ello es necesario utilizar el comando \textbf{\textbackslash ref} como se indica en la sección \ref{SEC:HIPERENLACES}. Muchos de los elementos vistos en este capítulo son entornos flotantes y es necesario tenerlo en cuenta a la hora de diseñar. Cuando corresponda se indicará si el elemento es flotante o no lo es.
