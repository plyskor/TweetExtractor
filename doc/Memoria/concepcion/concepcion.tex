Dados los requisitos anteriores, llega el momento de decidir la forma en que vamos a satisfacerlos. En primer lugar se enumeraran los distintos módulos que se desarrollarán, la concepción del modelo de datos  y las infraestructuras sobre las que se distribuirá el sistema.

\paragraph{Aplicaciones ejecutables}
Como nombre para nuestro sistema se ha elegido \textbf{TweetExtractor}, y contendrá dos aplicaciones:
\begin{itemize}
	\item \textbf{TweetExtractorFX:} Es la interfaz gráfica de usuario. Comprende un conjunto de ventanas a través de las cuales se pueden realizar diferentes actos de gestión de la aplicación como gestión de usuarios, extracciones y preferencias; visualización de resultados, reportes y gráficos...
	Se ha elegido JavaFX (se desarrolla en Java+XML) como librería para la creación de la interfaz gráfica debido a que ya viene integrada con el JDK y también a su sencillez, versatilidad e integración con el back-end en Java. Gracias a Java, podremos ejecutarla en todos los sistemas operativos que lo soporten.
	\item \textbf{TweetExtractor Server:} Es una aplicación web que funciona a modo de servidor. Sigue ejecutándose aunque la interfaz gráfica esté cerrada y se encarga de gestionar las tareas asíncronas (las crea, las modifica, controla su ejecución, las elimina... ). Se desarrollará en Java y se desplegará en un servidor de aplicaciones Apache Tomcat. Se comunicará con TweetExtractorFX a través de servicios web Java, que funcionarán sobre la infraestructura de los protocolos SOAP + HTTP + SSL/TLS. 
\end{itemize}
\paragraph{Entornos}
Como podemos observar en el diagrama inferior (\ref{FIG:ENVIRONMENTDESIGN}), podemos separar los entornos en 3:
\begin{itemize}
	\item \textbf{Entornos de usuario:} Son los entornos (que pueden ser distintos) donde se ejecutaría la interfaz gráfica. Debe tener conexión a Internet para conectarse a Twitter. Se limitarán los entornos compatibles para acotar la fase de pruebas: sólo se probará en Ubuntu (GNU/Linux), macOS y Microsoft Windows con una versión de Java superior a 1.8.
	\item \textbf{Entorno del servidor:} Es el entorno que más carga de memoria soportará. Debe tener conexión a Internet para conectarse a Twitter. Puede ser cualquier entorno donde se pueda desplegar un contenedor de aplicaciones web de Java. En nuestro caso se ha elegido un servidor Linux con una instancia de Apache Tomcat 9.0.21. Se podrá acceder desde Internet a nuestra instancia a través del nombre de dominio app.preciapps.com en el puerto 8080. \newpage
	\item \textbf{Entorno de almacenamiento:} Es el entorno que más carga de almacenamiento y conexiones de red soportará. En él se ejecutará una instancia de servidor de base de datos PostgreSQL. En nuestro caso, hemos optado también por un servidor Linux con una instancia PostgreSQL v11.3. Se podrá acceder desde Internet a nuestra instancia a través del nombre de dominio db.preciapps.com en el puerto 5432.
\end{itemize}
El diagrama siguiente trata de desglosar y hacer más fácilmente comprensible la configuración de los entornos y las interconexiones entre ellos. Nótese que los distintos entornos pueden también fusionarse (incluso funcionar en una única máquina), pero optaremos por no hacerlo por motivos de rendimiento (con los entornos separados la optimización de cada uno de ellos es más sencilla). 
\begin{figure}[Diseño del entorno]{FIG:ENVIRONMENTDESIGN}{Diagrama que muestra el diseño del entorno, con los diferentes móduos y las conexiones entre ellos.}
	\image{}{}{EnvironmentDesign}
\end{figure}
\newpage
