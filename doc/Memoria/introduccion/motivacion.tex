Tras lo primeros encuentros con el tutor, éste fue introduciéndonos y desarrollándonos una idea la cual consistía en una plataforma o interfaz donde existiese la posibilidad de seleccionar, extraer, analizar y tratar datos (desde distintas fuentes) sobre la participación ciudadana en las ciudades.

La idea sería que cada ``módulo'' de dicho sistema se dedicase a una de las posibles fuentes (por ejemplo, las redes sociales, el \href{https://datos.madrid.es/portal/site/egob/}{portal de datos abiertos de Madrid}) y que fuese el propio sistema el que se ocupase de la recopilación, organización y visualización de los resultados.

Como hemos indicado, entre las fuentes susceptibles de contener información objeto de análisis se encuentran las tan populares redes sociales. Concretamente, en la ciudad de Madrid existe la plataforma \href{https://decide.madrid.es/}{Decide Madrid}, donde se lanzan y votan propuestas e incluso se vota a qué proyectos se destina el dinero de los presupuestos municipales. Dicha plataforma tiene una cierta integración con Twitter, plataforma en la que los usuarios proponen iniciativas y también opinan y votan.

Con estas premisas y tras una pequeña investigación sobre la situación actual en este contexto, llegamos a la conclusión de que sería muy interesante el desarollo de un sistema como el que proponemos, que tenga como motivación hacer hincapié en estos puntos identitarios que le diferenciarían de los sistemas existentes hasta el momento:
\newpage
\begin{itemize}
	\item \textbf{Sistema integral:} El sistema permitirá el desarrollo de todas las acciones del proceso (extracción, guardado y gestión de datos, análisis, reportes...). 
	\item \textbf{Universal:} En paralelo a la integridad del sistema, los resultados deberían poder exportarse en formatos estándares para poder integrarlos fácilmente en un sistema externo
	\item \textbf{Versatilidad:} Con respecto al tema que es objeto de interés analítico, nuestro sistema debe ser lo suficientemente abstracto como para poder orientarlo y utilizarlo en dicho sentido.
	\item \textbf{Intuitividad y facilidad de uso:} Twitter está al alcance de todo el mundo, y por tanto sería interesante que nuestro sistema pudiera ser utilizado también por cualquier tipo de actor, apostando por la construcción de una interfaz gráfica de usuario. Los resultados deberían poderse consultar de una forma vistosa e intuitiva dentro del mismo sistema.
	\item \textbf{Adaptabilidad y escalabilidad:} El sistema debería de poder ser adaptable a nuevas necesidades o a necesidades específicas de un contexto concreto. 
\end{itemize}

