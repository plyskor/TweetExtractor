Tras lo primeros encuentros con el tutor, éste fue introduciéndonos y desarrollándonos una idea sobre la cual él había estado pensando. Consistía en una plataforma o interfaz donde existiese la posibilidad de seleccionar, extraer, analizar y tratar datos (desde distintas fuentes) sobre la participación ciudadana en las ciudades.

La idea sería que cada ``módulo'' de dicho sistema se dedicase a una de las posibles fuentes y que fuese el propio sistema el que se ocupase de la recopilación, organización y visualización de los resultados.

Entre las fuentes susceptibles de contener información objeto de análisis se encuentran las tan populares redes sociales. Concretamente, en la ciudad de Madrid existe la plataforma \href{https://decide.madrid.es/}{Decide Madrid}, donde se lanzan y votan propuestas e incluso se vota a qué proyectos se destina el dinero de los presupuestos municipales. Dicha plataforma tiene una cierta integración con Twitter, plataforma en la que los usuarios proponen iniciativas y también opinan y votan 
