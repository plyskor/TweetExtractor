% arara: clean: {files: [tfgtfmthesisuam.aux, tfgtfmthesisuam.idx, tfgtfmthesisuam.ilg, tfgtfmthesisuam.ind, tfgtfmthesisuam.bbl, tfgtfmthesisuam.bcf, tfgtfmthesisuam.blg, tfgtfmthesisuam.run.xml, tfgtfmthesisuam.fdb_latexmk, tfgtfmthesisuam.fls, tfgtfmthesisuam.loe, tfgtfmthesisuam.lof, tfgtfmthesisuam.lol, tfgtfmthesisuam.lot, tfgtfmthesisuam.ltb, tfgtfmthesisuam.out, tfgtfmthesisuam.toc, tfgtfmthesisuam.upa, tfgtfmthesisuam.upb, tfgtfmthesisuam.acn, tfgtfmthesisuam.acr, tfgtfmthesisuam.alg, tfgtfmthesisuam.glg, tfgtfmthesisuam.glo, tfgtfmthesisuam.gls, tfgtfmthesisuam.glsdefs, tfgtfmthesisuam.idx,  tfgtfmthesisuam.ilg, tfgtfmthesisuam.xdy, tfgtfmthesisuam.loa, tfgtfmthesisuam.gnuploterrors , tfgtfmthesisuam.mw, tfgtfmthesisuam.fdb_latexmk ]
% arara: pdflatex: {shell: yes}
% arara: makeglossaries
% arara: makeindex: {style: tfgtfmthesisuam.ist }
% !arara: bibtex
% arara: pdflatex: {shell: yes}
% arara: pdflatex: {shell: yes}
% arara: clean: {files: [tfgtfmthesisuam.aux, tfgtfmthesisuam.idx, tfgtfmthesisuam.ilg, tfgtfmthesisuam.ind, tfgtfmthesisuam.bbl, tfgtfmthesisuam.bcf, tfgtfmthesisuam.blg, tfgtfmthesisuam.run.xml, tfgtfmthesisuam.fdb_latexmk, tfgtfmthesisuam.fls, tfgtfmthesisuam.loe, tfgtfmthesisuam.lof, tfgtfmthesisuam.lol, tfgtfmthesisuam.lot, tfgtfmthesisuam.ltb, tfgtfmthesisuam.out, tfgtfmthesisuam.toc, tfgtfmthesisuam.upa, tfgtfmthesisuam.upb, tfgtfmthesisuam.acn, tfgtfmthesisuam.acr, tfgtfmthesisuam.alg, tfgtfmthesisuam.glg, tfgtfmthesisuam.glo, tfgtfmthesisuam.gls, tfgtfmthesisuam.glsdefs, tfgtfmthesisuam.idx,  tfgtfmthesisuam.ilg, tfgtfmthesisuam.xdy, tfgtfmthesisuam.loa, tfgtfmthesisuam.gnuploterrors , tfgtfmthesisuam.mw, tfgtfmthesisuam.fdb_latexmk ]}

\documentclass[epsbased,copyright,final,printable,covers,extendedindex,firstnumbered,tfg,gnuplot]{tfgtfmthesisuam}

\advisor{Iván Cantador Gutiérrez}
\levelin{Ingeniería Informática y Matemáticas}
\title{Herramienta de descarga y análisis de datos de Twitter sobre participación ciudadana}
\author{José Antonio García del Saz}
\privateaddress{C\textbackslash Las Torcas Nº1 Portal 3 2ºB}
\copyrightdate{20 de Junio de 2019}

\dedication{A mi familia, mis amigos, mis compañeros y mis maestros}
\famouscite{Si quieres aprender, enseña. \\[0.1em] \begin{flushright}Cicerón\end{flushright}}
\ackfile{inicio/agradecimientos}
\abstractfile{inicio/abstract}

\keywords{Algunas}
\palabrasclave{Otras}
\coverdata
{
  Escuela Politécnica Superior \\
  Universidad Autónoma de Madrid \\
  C\textbackslash Francisco Tomás y Valiente nº 11
}

\bibliographyconfig{tfgtfmthesisuam}

\datadir{data}
\graphicsdir{img}
\logosdir{img}
\codesdir{codes}

\begin{document}
\chapter{Introducción\label{CAP:INTRO}}{introduccion/introduccion}
	\section{Motivación del proyecto\label{SEC:MOTIVACION}}{introduccion/motivacion}
	\section{Objetivos y enfoque\label{SEC:OBJETIVOS}}{introduccion/objetivos}
\chapter{Participación ciudadana, teconología y redes sociales\label{CAP:PARTCIUDADANAYTEC}}{partytec/partytec}
		\section{Estado del Arte\label{SEC:ESTADOARTE}}{partytec/estadoarte}
		\section{Tecnologías utilizadas\label{SEC:TECUTILIZADAS}}{partytec/tecnologias}
\chapter{Análisis de requisitos\label{CAP:REQUISITOS}}{requisitos/requisitos}
	\section{Requisitos funcionales\label{SEC:REQFUNCIONALES}}{requisitos/funcionales}
	\section{Requisitos no funcionales\label{SEC:REQNOFUNCIONALES}}{requisitos/nofuncionales}
\chapter{Concepción y diseño\label{CAP:CONCEPCION}}{concepcion/concepcion}
	\section{Modelo de datos\label{SEC:MODELODATOS}}{concepcion/modelodatos}
\chapter{Desarrollo y mantenimiento\label{CAP:DESYMANT}}{desymant/desymant}
	\section{Mantenimiento\label{SEC:MANTENIMIENTO}}{desymant/mantenimiento}
\chapter{Resultados y ejemplos\label{CAP:RESULTADOS}}{resultados/resultados}
\chapter{Conclusiones y trabajo futuro\label{CAP:CONCLUSIONES}}{conclusiones/conclusiones}


\appendix

\chapter{Diagramas\label{CAP:DIAGRAMAS}}{anexos/diagramas}
\chapter{Tablas\label{CAP:TABLAS}}{anexos/tablas}
\chapter{Códigos\label{CAP:CODIGOS}}{anexos/codigos}
\chapter{Imágenes y capturas de la interfaz gráfica\label{CAP:GUICAPTURES}}{anexos/guicaptures}

\end{document}
