La expansión de las nuevas redes sociales despierta un interés analítico que pretende extraer conclusiones sobre la forma que tiene el ser humano de comunicarse con otros sujetos a través de Internet.

En este proyecto vamos a modelar y desarrollar un sistema integral y adaptable para extraer datos de Twitter usando la API REST que dicha red social nos proporciona.
Este sistema ofrecerá al usuario la opción de acotar los datos relevantes, extraerlos, almacenarlos y depués obtener diversos tipos de análisis y reportes que le lleven a las respuestas deseadas.

Las características que diferenciará a nuestro sistema serán la alta versatilidad, la facilidad en el escalado, la sencillez en la utilización y comprensión de los datos y la alta integrabilidad.

Como caso de uso concreto, se evaluará el sistema en el contexto e la participación ciudadana en Madrid (con la plataforma Decide Madrid).

\palabrasclave{Redes Sociales, Twitter, Análisis, Extracción, Participación Ciudadana }
