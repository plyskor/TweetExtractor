En esta tabla se enumeran las dependencias externas de nuestro sistema:

\begin{table}[Tecnologías utilizadas]{TB:TECNOLOGIAS}{En esta tabla se citan las tecnologías utilizadas en el proyecto.}
  \begin{tabular}{|c|c|p{11cm}|}
    \hline
    \textbf{Nombre} & \textbf{Versión} & \textbf{Objetivos} \\
    \hline \hline
    Twitter API & v4.0 &Esta API REST nos la proporciona Twitter para acceder a sus datos. Con ella realizaremos consultas que nos permitirán obtener los datos que son sujeto de nuestro análisis \\
    OpenJDK& v12.0.1 &La versión OpenSource del más que conocido Java Developpement Kit. Nuestra aplicación se desarrollará en Java y el entorno gráfico lo gestionará JavaFX. El servidor es un proyecto Java EE\\
    Twitter4J & v4.0.7 & Una librería Java que ofrece métodos y clases para la explotación de la API de Twitter en el código Java\\
    Spring Framework & v5.1.7 & Framework de código abierto para el desarrollo de aplicaciones y contenedor de inversión de control. Nos permitirá compartir recursos entre los diferentes puntos de la aplicación a través de contextos. \\
	Hibernate & v5.4.3 & Herramienta de mapeo objeto-relacional que se apoyará sobre el driver jdbc para conectar los módulos de nuestra aplicación al servidor de bases de datos. \\
	PostgreSQL Server & v11.3 & Servidor de bases de datos que guardará y gestionará todos nuestros datos.\\
	Apache Tomcat & v8.5.41  & Servidor de aplicaciones Java donde estará desplegado el módulo servidor de nuestra aplicación. \\
	Kumo API & v1.17 & Librería Java que nos permite crear Word Clouds muy configurables a partir de palabras.\\
	JFreeChart & v1.0.19 & Librería para generar gráficos de distintos tipos en código Java. Se usará para mostrar resultados de los análisis. \\
	Apache Lucene & v8.0.0 & Librería para la recuperación de información desde texto y web para el código Java. Se usará para la tokenización y tratamiento de textos.\\
	SSL/TLS & v1.2 & Protocolo criptográfico que garantiza las conexiones seguras en la red. Se implementará en todas y cada una de las comunicaciones que se realicen entre cada módulo de nuestra aplicación.\\
    \hline
  \end{tabular}
\end{table}