Con la revolución tecnológica ha cambiado de forma sustancial la forma en que los ciudadanos interactuamos con nuestras ciudades y con nuestros vecinos y gobernantes. La explosión de las redes sociales (como Twitter) ha provocado que aparezca la posibilidad de interpelar de forma directa (y también pública) a empresas, entidades púbicas, individuos, etc, con lo que la diversidad de opiniones públicas y la participación ciudadana abogan por transformar nuestras democracias representativas en democracias más directas como las que ya existieron en Atenas o la República Romana, o como las existen hoy en Suiza.

Es por esto que han nacido proyectos como \href{http://consulproject.org/en/}{CONSUL}, con los cuales se implementan interfaces que facilitan la participación ciudadana online a través de foros de opinión,de votaciones de propuestas on-line o de herramientas que se apadtan a cada ciudad (a sus distritos, barrios, comunidades, etc.) 

\begin{figure}[Consul en España y el mundo]{FIG:CONSUL}{Ayuntamientos que utilizan el proyecto CONSUL para participación ciudadana.}
	\subfigure[SBFIG:CONSULSPAIN]{Ayuntamientos en España}{\image{3cm}{}{consulSpain}} \quad
	\subfigure[SBFIG:CONSULWORLDWIDE]{Ayuntamientos en el mundo}{\image{3cm}{}{consulWorldwide}}
\end{figure}

Este tipo de herramienta ya está siendo utilizada por 33 países, 100 instituciones y 90 millones de ciudadanos en todo el mundo. En concreto en la ciudad de Madrid funciona desde 2015 bajo el nombre de \href{https://decide.madrid.es/}{Decide Madrid} y está notablemente integrada con la red social Twitter (pueden compartirse  a través de ella propuestas, apoyos, opiniones...).

Este tipo de herramientas, junto con las versátiles API de Twitter proporcionan la posibilidad de navegar en un universo de datos de los cuáles se pueden formular hipótesis sobre temas tales como qué preocupa a los ciudadanos, cuáles son las medidas más o menos populares, qué usuarios son más activos y cómo se relacionan entre ellos, etc. Y todo esto en tiempo real.

