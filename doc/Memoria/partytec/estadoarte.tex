\paragraph{Extracción y análisis de datos en redes sociales}
Las redes sociales modernas han cambiado para siempre la forma en que se comunican las personas. Los datos que hay presentes en ellas se han convertido por este motivo en un conjunto muy suculento de información para los investigadores. 

Un ejemplo de estas redes sociales es el servico de microblogging Twitter. Twitter es una red social con topología de grafo dirigido: un usuario puede seguir a otro cuyos tweets le interesan, pero esto no implica que el otro usuario le tenga que seguir a él (o dicho de otra manera, que le interesen sus tweets). Esto supone que los usuarios no usan tanto esta red para comunicarse con  sus amigos (como en Facebook por ejemplo), sino más bien para seguir a personas cuyas palabras les resultan interesantes (aunque no se conozcan).

El crecimiento desenfrenado que experimentó esta red social desde su lanzamiento en 2007 la han convertido en una de las más utilizadas del mundo, lo que la hace muy útil para detectar y monitorizar eventos del mundo real en directo.

Este interés analítico ha dado lugar al nacimiento de herramientas para la extracción y tratamiento de datos en Twitter tales como TwitterEcho \cite{Bosnjak2012}. Para el desarrollo de esta herramienta se tuvieron muy en cuenta las limitaciones que tiene la API proporcionada directamente por Twitter para las consultas (aparte de prohibir explícitamente el que se compartan los datos extraídos ni siquiera con fines académicos hay que sumar que la API de Twitter tiene una tasa de limitación para el número de consultas por unidad de tiempo que puede realizar un usuario), por lo que se apuesta por extraer los Tweets con un crawler HTTP desde la propia página web de Twitter (lo que resulta en saltarse la tasa de limitación de la API). Aunque sería una herramienta muy interesante para nuestros objetivos, está limitada a los tweets en portugués.

También han surgido estudios que tratan de averiguar cuál es el mejor método para extraer tweets verdaderamente relevantes sobre un tema deseado \cite{Criado2013}. En este artículo se muestra cómo a través de un sencillo algoritmo llamado ``algoritmo de adaptación de palabras clave refinado'', se pueden obtener más tweets relevantes y menos ruido que lanzando consultas con otros métodos como las básicas por palabras clave . Muy generalizadamente, este algoritmo comenzaría extrayendo tweets con una consulta básica de palabras clave e iría añadiendo o quitando estas palabras clave ayudándose de los hashtags que aparecen con frecuencia en los tweets, afinando así la búsqueda. Esto pone de relevancia lo interesante que es poder acotar las extracciones que se realizan para obtener la información que es relevante para el analista.

Por último, cabe destacar propuestas como esta \cite{Noordhuis2010}, en la que se intentan paliar los altos costes que conlleva el análisis de estos datos derivados en gran medida de la cantidad de datos y por ende de la cantidad de recursos necesarios para almacenarlos y analizarlos. Para hacerlo, se apuesta por los recursos computacionales en la nube, en concreto Amazon Web Services, y se ofrece un caso de uso en el que se aplica el algoritmo PageRank a las cuentas de Twitter en función de sus followers/followees.

\paragraph{Redes sociales: Gobernanza Electrónica y participación ciudadana}
Con la revolución tecnológica ha cambiado de forma sustancial la forma en que los ciudadanos interactuamos con nuestras ciudades y con nuestros vecinos y gobernantes. La explosión de las redes sociales (como Twitter) ha provocado que aparezca la posibilidad de interpelar de forma directa (y también pública) a empresas, entidades púbicas, individuos, etc, con lo que la diversidad de opiniones públicas y la participación ciudadana abogan por transformar nuestras democracias representativas en democracias más directas como las que ya existieron en Atenas o la República Romana, o como las existen hoy en Suiza.

Es por esto que han nacido proyectos como \href{http://consulproject.org/en/}{CONSUL}, con los cuales se implementan interfaces que facilitan la participación ciudadana online a través de foros de opinión,de votaciones de propuestas on-line o de herramientas que se apadtan a cada ciudad (a sus distritos, barrios, comunidades, etc.) 

\begin{figure}[CONSUL en España y el mundo]{FIG:CONSUL}{Ayuntamientos que utilizan el proyecto CONSUL para participación ciudadana.}
	\subfigure[SBFIG:CONSULSPAIN]{Ayuntamientos en España}{\image{3cm}{}{consulSpain}} \quad
	\subfigure[SBFIG:CONSULWORLDWIDE]{Ayuntamientos en el mundo}{\image{3cm}{}{consulWorldwide}}
\end{figure}

Este tipo de herramienta ya está siendo utilizada por 33 países, 100 instituciones y 90 millones de ciudadanos en todo el mundo. En concreto en la ciudad de Madrid funciona desde 2015 bajo el nombre de \href{https://decide.madrid.es/}{Decide Madrid} y está notablemente integrada con la red social Twitter (pueden compartirse  a través de ella propuestas, apoyos, opiniones...).

Este tipo de herramientas, junto con las versátiles API de Twitter proporcionan la posibilidad de navegar en un universo de datos de los cuáles se pueden formular hipótesis sobre temas tales como qué preocupa a los ciudadanos, cuáles son las medidas más o menos populares, qué usuarios son más activos y cómo se relacionan entre ellos, etc. Y todo esto en tiempo real.

