


\begin{functional}

\paragraph{Autentificación}
        \item El acceso al sistema está restringido para usuarios registrados.
        \begin{functional}
                \item El registro es libre, pudiendo hacerse desde el propio sistema.
                \item Las credenciales se componen de un nombre de usuario (único) y contraseña
        \end{functional}
        \item Las contraseñas tendrán entre 6-16 caracteres y contendrá al menos una mayúscula, una minúscula y un número.
        \item Un usuario que ha iniciado sesión puede cambiar su contraseña desde la GUI para los accesos posteriores.
        \item Un usuario que ha iniciado sesión puede eliminar su cuenta, eliminando también todos los datos que hubiere almacenado en la base de datos (extracciones, reportes, gráficos, etc.).

\paragraph{Extracciones}

	\item Un usuario puede crear extracciones desde la GUI que serán de su propiedad.
	\item El perímetro de una extracción se delimita (durante la creación) a través de filtros, necesarios para la creación de la extracción (cada extracción contiene al menos un filtro).
	\item  Un usuario puede alimentar una extracción en primer plano (desde la GUI) o en segundo plano (con una tarea asíncrona del servidor).
	\item Una extracción podrá alimentarse de forma indefinida en segundo plano.
	\item Los tweets extraídos contenidos en una extracción pertenecen a ésta: si se elimina la extracción se eliminan los tweets.
	\item Los tweets de las extracciones son consultables en crudo desde la GUI. También pueden eliminarse de forma individual desde la GUI.
	\item Los tweets de una extracción se pueden exportar en un fichero XML para su integración externa.
	\item Una misma extracción no podrá contener dos veces el mismo tweet.
	
\paragraph{Credenciales de la API de Twitter}

\item Un usuario puede añadir, modificar y eliminar credenciales para la API de Twitter en su cuenta.
\item El usuario debe tener al menos unos credenciales añadidos para poder comenzar a crear extracciones, tareas y reportes analíticos.

\paragraph{Tareas asíncronas}

\item Un módulo servidor web existirá para gestionar la ejecución de tareas asíncronas en segundo plano.
\item La conexión entre la GUI y el servidor es configurable desde la GUI, y esta configuración se guarda en el registro del sistema operativo. 
\item La GUI se comunicará con el módulo servidor a través de servicios web (SOAP sobre SSL/TLS).
\item Las tareas asíncronas tienen un ciclo de vida específico que depende del tipo de tarea. (Ver diagrama \ref{FIG:SERVERTASKLIFECYCLE})
\item Con el arranque del servidor, las tareas existentes son cargadas desde la base de datos.
\item Existen tipos de tareas que se ejecutan de forma indefinida si nunca son detenidas
\item El usuario puede crear, eliminar, preparar, lanzar y parar tareas asíncronas desde la GUI.
\item Una tarea puede ser programada para ejecutarse en un momento dado del futuro.
\item Tras un reinicio del servidor, las tareas programadas aún sin caducar deben ser reprogramadas automáticamente, las tareas que se estaban ejecutando deberán volver a ejecutarse automáticamente y las tareas que hayan caducado pasarán a dicho estado.
\item La ejecución de una tarea programada puede ser cancelada pasando dicha tarea al estado "preparada".

\paragraph{Análisis}

\item Un usuario puede crear, modificar y eliminar diversos tipos de reportes analíticos sobre los datos extraídos.
\item Los contenidos (registros) de un reporte pueden ser actualizados para no quedar obsoletos con el tiempo.
\item Un reporte puede ser actualizado en segundo plano con una tarea asíncrona de servidor.
\item Los datos de un reporte se pueden consultar en crudo en la GUI y exportarse en un archivo .csv para su integración externa.
\item Se guardarán en base de datos tanto el instante en que se creó el reporte como el instante en el que se actualizó por última vez.
\item Existen reportes sobre la evolución del volumen de tweets en el tiempo. 
\item Existen reportes sobre las tendencias en las extracciones (hashtags, usuarios, menciones, términos...).
\item Existen reportes sobre la clasificación semántica de los tweets.
\paragraph{Gráficos}
\item Desde la GUI, un usuario puede crear, configurar, modificar y eliminar distintos tipos de gráficos que muestran los datos de los reportes disponibles.
\item Cada tipo de gráfico tiene unas configuraciones específicas (tipos de línea, colores, tamaño y estilo de fuentes, etc.) que serán parametrizables desde la GUI durante la creación del gráfico. 
\item Las configuraciones de cada tipo de gráfico se guardan en base de datos para ser reutilizadas posteriormente.
\item Los gráficos pueden tanto verse desde la GUI como exportarse a un archivo JPEG.
\item Existe una matriz de compatibilidad entre los tipos de reportes y los tipos de gráficos que son compatibles entre sí:
\begin{figure}[Matriz de compatibilidad reporte-gráfico]{FIG:COMPMATRIXREPORTCHRART}{Matriz de compatibilidad entre los tipos de reporte analítico y los tipos de gráficos.}
	\image{5cm}{}{CompatibilityMatrixReportChart}
\end{figure}
\paragraph{Procesamiento del lenguaje natural (NLP)}
\item Para cada idioma, un usuario puede crear listas de palabras ignorables en ese idioma. De esta manera se permite personalizar las palabras irrevelantes en cada contexto y en cada idioma.
\item Las palabras ignorables no se tendrán en cuenta para la medición de frecuencias o la elaboración de reportes relacionados con dichas frecuencias.
\item Para cada idioma, el usuario podrá crear, modificar y eliminar diferentes preferencias para el procesamiento del lenguaje natural desde la GUI. Dichas preferencias contienen una serie de categorías y subcategorías (temas) a través de las cuales luego se podrán clasificar los tweets.
\item Dado un conjunto de extracciones, un usuario puede separar todos sus tweets en palabras para obtener un conjunto de términos presentes en los tweets.
\item Los conjuntos de términos pueden crearse y eliminarse desde la GUI y se almacenan en la base de datos.
\item Dado un conjunto de términos y unas preferencias NLP, el usuario puede clasificar los términos en las diferentes categorías y subcategorías desde la GUI. De este modo se entrena la clasificación semántica de tweets en cada contexto.
\item La clasificación de los términos se puede hacer en paralelo (varios humanos usando la aplicación en varias máquinas) sin que se clasifique la misma palabra dos veces ni en serie ni en paralelo.


\end{functional}
