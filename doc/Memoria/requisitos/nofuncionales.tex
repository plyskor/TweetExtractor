\begin{nonfunctional}
\paragraph{Entorno}
\item El sistema es distribuido, siendo sus módulos aislables y orientables.
\item Las conexiones con la base de datos no se realizan en texto plano. Debe cifrarse punto a punto mediante SSL/TLS.
\item Los diferentes módulos generan ficheros de log para el mejor trazado de las actividades.
\item Los logs históricos se comprimen para optimizar el uso de almacenamiento. 
\item El sistema debe proporcionar mensajes de error que sean informativos y orientados a usuario final.
\item Las tareas largas (acceso a datos, análisis, generación de reportes...) no bloquean la GUI completamente: se ejecutan en otro hilo mientras se muestra una ventana modal que informa de las acciones que ocurren en paralelo.
\paragraph{Autentificación}
 \item Las contraseñas de los usuarios no se pueden guardar en texto plano en la base de datos.
 \paragraph{Extracciones}
 \item Una extracción no puede ser alimentada desde dos lugares distintos al mismo tiempo, ni siquiera desde la misma máquina.
 \item Se tendrán en cuenta las restricciones para las credenciales gratuitas que tiene la API de Twitter. 
 \paragraph{Tareas asíncronas}
 \item Una tarea no puede ejecutarse varias veces en paralelo, sólo se concibe una ejecución a la vez por tarea.
 \item Ningún intercambio de datos entre la GUI y el servidor se puede hacer en texto plano. Siempre se debe usar el cifrado punto a punto sobre SSL/TLS.
\end{nonfunctional}