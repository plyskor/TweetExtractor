\paragraph{Control de versiones. Compilación}
Como se mencionó al principio de este capítulo, tanto el código que compone nuestro sistema como esta memoria se encuentran publicadas en \href{https://github.com/plyskor/TweetExtractor}{GitHub}. 

En dicho repositorio pueden encontrarse los evolutivos y las correcciones de defectos que se iban descubriendo durante las pruebas para las ejecuciones de los casos de uso.
 
 Una vez clonado el código, la compilación y generación de binarios es muy sencilla: sólo necesitamos una máquina donde estén instalados tanto el JDK de Java (v1.8+) como la herramienta Maven. Basta con lanzar el comando ``mvn clean install'' desde el directorio raíz para que se realize todo el proceso de compilado y empaquetado.
 
\paragraph{SonarQube}
Dado la cantidad de código que comenzaba a juntarse durante el desarrollo, se optó por utilizar una herramienta de inspección continua de código como \href{https://www.sonarqube.org/}{SonarQube}. 

Se ha desarrollado una tarea programada cada noche que descarga de GitHub la útima versión del código, la compila y envía los binarios a SonarQube. Tras un análisis exhaustivo, SonarQube nos ofrece distintos reportes sobre nuestro código. Aquí un ejemplo tras analizar nuestro proyecto:

\begin{figure}[Reporte general SonarQube]{FIG:SONARGENERAL}{Resumen del análisis de TweetExtractor. Se muestra el número de bugs, las vulnerabilidades, la hediondez del código, la cobertura de tests automáticos y el código duplicado. También se muestra el conteo de líneas de código (25.000 entre Java y XML)}
	\image{}{}{sonar/generalReport}
\end{figure}

Un conjunto mas específico y desglosado de las medidas se puede encontrar en esta imagen de los apéndices (ver \ref{FIG:SONARMEASURES}). 

Pero además de detectar los errores, SonarQube es capaz de encontrar el problema concreto en cada fichero de código fuente y ofrecer de una manera muy didáctica sugerencias para solucionarlo y mejorar la robustez del sistema:

\begin{figure}[Sugerencia mantenimiento SonarQube]{FIG:SONARCODEALERT}{Advertencia sobre vulnerabilidad. Se nos recomienda devolver una lista vacía (en lugar de un ``null'') en caso de no encontrar resultados en este método del DAO.}
	\image{}{}{sonar/sonarBuggedCode}
		\image{}{}{sonar/sonarAlert}
\end{figure}
\newpage
Sobre cada uno de los apartados que hemos visto en \ref{FIG:SONARMEASURES}, los resultados específicos pueden mostrarse de formas muy diversas, como esta gráfica sobre la seguridad del sistema clase a clase:

\begin{figure}[Reporte sobre seguridad de SonarQube]{FIG:SONARSECURITY}{Gráfica que muestra la seguridad del código clase a clase. Incluye una estimación del tiempo necesario para solucionar los eventuales problemas.}
	\image{}{}{sonar/sonarSecurityReport}
\end{figure}
