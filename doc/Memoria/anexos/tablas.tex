\begin{table}[Idiomas soportados por la API de Twitter]{TB:IDIOMASAPI}{En esta tabla se muestran los idiomas que la API de Twitter es capaz de identificar. Serán los idiomas para los que funcione nuestro análisis semántico}
	
	\subtable{STB:IDIOMASAPI1}{Idiomas (I)}
	{
		\begin{tabular}{|c|c|}
		\hline
		\textbf{Idioma} & \textbf{Código} \\
		French & fr\\ 
		English & en\\ 
		Arabic & ar\\ 
		Japanese & ja\\ 
		Spanish & es\\ 
		German & de\\ 
		Italian & it\\ 
		Indonesian & id\\ 
		Portuguese & pt\\ 
		Korean & ko\\ 
		Turkish & tr\\ 
		Russian & ru\\ 
		Dutch & nl\\ 
		Filipino & fil\\ 
		Msa & msa\\ 
		Zh-tw & zh-tw\\ 
		Zh-cn & zh-cn\\ 
		Hindi & hi\\ 
		Norwegian & no\\ 
		Swedish & sv\\ 
		Finnish & fi\\ 
		Danish & da\\ 
		Polish & pl\\ 
		\hline \hline
		\hline
		\end{tabular}
	}
	\subtable{STB:IDIOMASAPI2}{Idiomas (II)}
	{
		\begin{tabular}{|c|c|}
		\hline
		\textbf{Idioma} & \textbf{Código} \\
		Hungarian & hu\\ 
		Persian & fa\\ 
		Hebrew & he\\ 
		Urdu & ur\\ 
		Thai & th\\ 
		Ukrainian & uk\\ 
		Catalan & ca\\ 
		Irish & ga\\ 
		Greek & el\\ 
		Basque & eu\\ 
		Czech & cs\\ 
		Gallegan & gl\\ 
		Romanian & ro\\ 
		Croatian & hr\\ 
		En-gb & en-gb\\ 
		Vietnamese & vi\\ 
		Bengali & bn\\ 
		Bulgarian & bg\\ 
		Serbian & sr\\ 
		Slovak & sk\\ 
		Gujarati & gu\\ 
		Marathi & mr\\ 
		Tamil & ta\\ 
		Kannada & kn\\
		\hline \hline
		\hline
		\end{tabular}
	}

\end{table}

\begin{table}[Referencia de servicios web SOAP de TweetExtractorServer]{TB:SOAPWSREFERENCE}{En esta tabla se muestran los diferentes servicios web SOAP disponibles en el módulo servidor, así como una breve descripción sobre su funcionalidad y un hipervínculo a sus descriptores WSDL accesibles desde HTTPS}
	\begin{tabular}{|p{10cm}|p{7cm}|}
	\hline
	\textbf{URL} & \textbf{Comentario} \\
	\href{https://app.preciapps.com:8080/tweetextractor-server-1.0.1.0/createServerTaskTimelineTopNHashtagsReportImpl?wsdl}{https://app.preciapps.com:8080/tweetextractor-server-1.0.1.0/ createServerTaskTimelineTopNHashtagsReportImpl} & Crea una tarea  ServerTaskTimelineTopNHashtagsReport\\
	\href{https://app.preciapps.com:8080/tweetextractor-server-1.0.1.0/createServerTaskTimelineVolumeReport?wsdl}{https://app.preciapps.com:8080/tweetextractor-server-1.0.1.0/ createServerTaskTimelineVolumeReport} & Crea una tarea ServerTaskTimelineVolumeReport \\
	\href{https://app.preciapps.com:8080/tweetextractor-server-1.0.1.0/createServerTaskTrendsReport?wsdl}{https://app.preciapps.com:8080/tweetextractor-server-1.0.1.0/ createServerTaskTrendsReport} &  Crea una tarea ServerTaskTrendsReport\\
	\href{https://app.preciapps.com:8080/tweetextractor-server-1.0.1.0/createServerTaskTweetVolumeByNERTopicsReport?wsdl}{https://app.preciapps.com:8080/tweetextractor-server-1.0.1.0/ createServerTaskTweetVolumeByNERTopicsReport} & Crea una tarea  ServerTaskTweetVolumeByNERTopicsReport\\
	\href{https://app.preciapps.com:8080/tweetextractor-server-1.0.1.0/createServerTaskTweetVolumeByNamedEntitiesReport?wsdl}{https://app.preciapps.com:8080/tweetextractor-server-1.0.1.0/ createServerTaskTweetVolumeByNamedEntitiesReport} &  Crea una tarea ServerTaskTweetVolumeByNamedEntitiesReport \\
	\href{https://app.preciapps.com:8080/tweetextractor-server-1.0.1.0/createServerTaskUpdateExtractionIndef?wsdl}{https://app.preciapps.com:8080/tweetextractor-server-1.0.1.0/ createServerTaskUpdateExtractionIndef} & Crea una tarea ServerTaskUpdateExtractionIndef \\
	\href{https://app.preciapps.com:8080/tweetextractor-server-1.0.1.0/deleteServerTask?wsdl}{https://app.preciapps.com:8080/tweetextractor-server-1.0.1.0/ deleteServerTask} &  Borra una tarea y todos sus datos\\
	\href{https://app.preciapps.com:8080/tweetextractor-server-1.0.1.0/getServerStatus?wsdl}{https://app.preciapps.com:8080/tweetextractor-server-1.0.1.0/ getServerStatus} & Se usa a modo de ping. \\
	\href{https://app.preciapps.com:8080/tweetextractor-server-1.0.1.0/getServerTaskStatus?wsdl}{https://app.preciapps.com:8080/tweetextractor-server-1.0.1.0/ getServerTaskStatus} & Devuelve el estado de una tarea \\
	\href{https://app.preciapps.com:8080/tweetextractor-server-1.0.1.0/getUserServerTasks?wsdl}{https://app.preciapps.com:8080/tweetextractor-server-1.0.1.0/ getUserServerTasks} & Devuelve una lista con datos te todas las tareas \\
	\href{https://app.preciapps.com:8080/tweetextractor-server-1.0.1.0/interruptServerTask?wsdl}{https://app.preciapps.com:8080/tweetextractor-server-1.0.1.0/ interruptServerTask} &  Interrumpe una tarea ejecutándose\\
	\href{https://app.preciapps.com:8080/tweetextractor-server-1.0.1.0/launchServerTask?wsdl}{https://app.preciapps.com:8080/tweetextractor-server-1.0.1.0/ launchServerTask} &  Lanza una tarea ``Preparada''\\
	\href{https://app.preciapps.com:8080/tweetextractor-server-1.0.1.0/scheduleServerTask?wsdl}{https://app.preciapps.com:8080/tweetextractor-server-1.0.1.0/ scheduleServerTask} & Programa una tarea \\
	\href{https://app.preciapps.com:8080/tweetextractor-server-1.0.1.0/setServerTaskReady?wsdl}{https://app.preciapps.com:8080/tweetextractor-server-1.0.1.0/ setServerTaskReady} & Pasa una tarea al estado ``Preparada'' \\
	\hline \hline
	\hline
	\end{tabular}
\end{table}
\begin{table}[Stackholders utilizados para las extracciones]{TB:STACKHOLDERS}{En esta tabla se muestran las cuentas de Twitter más relacionadas naturalmente con el tipo de tweets que se busca obtener}
		\begin{tabular}{|p{4cm}|p{8cm}|}
			\hline
			\textbf{Cuenta de Twitter} & \textbf{Comentario} \\
			@AhoraMadrid & Partido Ahora Madrid\\
			@Cs\_Madrid & Partido Ciudadanos Regional\\
			@CsMadridCiudad & Partido Ciudadanos Municipal\\
			@DecideMadrid & Plataforma de participación ciudadana Decide Madrid \\
			@GrupoPPMadrid & Partido popular municipal \\
			@JMDArganzuela & Junta de distrito de Arganzuela\\
			@JMDBarajas & Junta de distrito de Barajas\\
			@JMDCarabanchel & Junta de distrito de Carabanchel\\
			@JMDCentro & Junta de distrito Centro\\
			@JMDChamartin & Junta de distrito Chamartín \\
			@JMDChamberi & Junta de distrito Chamberi\\
			@JMDCiudadLineal & Junta de distrito Ciudad Lineal \\
			@JMDFuencarral & Junta de distrito Fuencarral\\
			@JMDHortaleza & Junta de distrito Hortaleza \\
			@JMDLatina & Junta de distrito Latina\\
			@JMDmoratalaz & Junta de distrito Moratalaz\\
			@JMDpvallecas & Junta de distrito Vallecas\\
			@JMDretiro & Junta de distrito Retiro\\
			@JMDSalamanca & Junta de distrito Salamanca\\
			@JMD\_SanBlas & Junta de distrito San Blas \\
			@JMDTetuan & Junta de distrito Tetuán \\
			@jmd\_usera & Junta de distrito Usera\\
			@jmdvicalvaro & Junta de distrito Vicálvaro\\
			@JMD\_villaverde & Junta de distrito de Villaverde\\
			@JMDvivallecas & Junta de distrito Villa de Vallecas\\
			@LineaMadrid & Cuenta de servicios de atención al ciudadano\\
			@MADRID & Cuenta oficial del Ayuntamiento de Madrid\\
			@MoncloaAravaca & Junta de distrito Moncloa-Aravaca\\
			@PPAsamblea & Partido popular regional\\
			@psoe\_m & Partido Socialista Obrero Español Regional\\
			@psoemadridayto & Partido Socialista Obrero Español Municipal\\
			\hline \hline
			\hline
		\end{tabular}
\end{table}

\begin{table}[Relación de categorías y subcategorías para análisis semántico]{TB:NAMEDENTITIESTOPICS}{En esta tabla se muestran las categorías y subcategorías declaradas para el análisis semántico de los tweets}
	\begin{tabular}{|p{5cm}|p{12.5cm}|}
		\hline
		\textbf{Categoría (Named Entity)} & \textbf{Subcategorías (Topics)} \\
		Accesibilidad &  Barreras arquitectonicas, Movilidad reducida, Discapacitados, Aparcamientos para discapacitados, Invidentes \\
		Animales &  Fauna urbana, Mascotas, Animales callejeros, Derechos de los animales, Sanidad animal, Plagas, Perros, Areas caninas, Adiestramiento canino, Felinos, Palomas urbanas, Ganaderia, Excrementos de animales, Castracion de animales \\
		Asociaciones &  Comunidades de vecinos, ONG's, Asociacionismo \\
		Ayuntamiento y administración pública &  Recortes, Servicios publicos, Remunicipalizacion, Respeto y civismo, Wi-Fi, Ordenanza municipal, Funcion publica, Gestion publica, Administracion publica, Barrios, Equipo de gobierno \\
		Civismo &  Respeto y civismo, Respeto por los mayores, Respeto por los menores, Descanso vecinal, Convivencia \\
		Cultura &  Educacion y cultura, Centros culturales, Servicios culturales, Museos, Monumentos, Recintos feriales, Artistas, Arte urbano, Artesania, Literatura, Tauromaquia, Planetario, Guerra civil \\
		Delicuencia &  Delitos, Atracos, Robos, Estafas, Especulacion, Corrupcion, Terrorismo, Malos tratos, Bullying, Vandalismo, Botellon, Graffitis, Prostitucion, Multas \\
		Deportes &  Deportes, Instalaciones deportivas, Clubs deportivos, Eventos deportivos, Futbol, Baloncesto, Atletismo, Natacion, Ciclismo, Triatlon, Automovilismo, Golf, Baile, Escalada, Patinaje, Patinetes, Gimnasia ritmica, Aparatos de gimnasia, Gimnasios \\
		Derechos sociales &  Subvenciones publicas, Bienestar ciudadano, Pobreza, Apoyo emocional, Derecho a la informacion, Ayudas sociales, Ayudas economicas, Servicios sociales, Derechos humanos, Derechos sociales y educacion \\
		Economía &  Actividad economica, Emprendimiento, Negocios, Comercio, Centros comerciales, Bancos, Impuestos, Impagos, Ingresos, Inversiones, Economia social, Dinero publico, Empresas \\
		\hline \hline
		\hline
	\end{tabular}
\end{table}


\begin{table}[Relación de categorías y subcategorías para análisis semántico (II)]{TB:NAMEDENTITIESTOPICS2}{Segunda parte de la tabla donde se muestran las categorías y subcategorías declaradas para el análisis semántico de los tweets.}
	
	\begin{tabular}{|p{4.5cm}|p{12cm}|}
		\hline
		\textbf{Categoría (Named Entity)} & \textbf{Subcategorías (Topics)} \\
		Educación &  Educacion publica, Formacion, Educacion sexual, Investigacion, Fracaso escolar, Bibliotecas, Ludotecas, Profesores, Universidades, Institutos, Escuelas de musica, Idiomas, Filosofia, Colegios, Educacion infantil \\
		Empleo &  Derechos laborales, Trabajos precarios, Desempleo, Empleo publico, Empleo sostenible, Teletrabajo, IRPF, Empleo \\
		Equidad e integración &  Igualdad, Igualdad de genero, Inclusion social, Discriminacion, Concienciacion, Solidaridad, Personas sin hogar, Inmigracion, Exilio y refugio, Violencia de genero, Ciudad amable, Chabolas, Mujeres, LGTBI \\
		Familia e infancia &  Conciliacion familiar, Maternidad, Infancia, Proteccion del menor, Divorcio, Zonas infantiles, Familias \\
		Jóvenes &  Juventud, Adolescencia \\
		Justicia &  Memoria historica, Legislacion, Legalizacion, Leyes, Libertades, Libertad de expresion, Regulacion, Ley anti tabaco, Delitos fiscales, Defensa juridica \\
		Medio ambiente &  Contaminacion acustica, Contaminacion luminica, Contaminacion odorifera, Vibraciones, Radioactividad, Vehiculos electricos, Salubridad publica, Agricultura, Ecologismo, Arboles, Electricidad, Calefaccion, Contaminacion atmosferica, Servicios de limpieza, Parques y jardines, Zonas verdes, Contenedores, Basuras, Medio ambiente, Medio ambiente y limpieza \\
		Movilidad &  Movilidad y educacion, Restricciones de trafico, Informacion del trafico, Limites de velocidad, Aparcamiento regulado, Aparcacoches, Abono de transportes, Tranvia y metro ligero, Trenes, Impuestos sobre vehiculos, Peajes, Motocicletas, Conductores, Carga y descarga, Accesibilidad peatonal, Movilidad y seguridad, Metro, Autobuses, Movilidad en bicicleta, Vehiculos, Carril bus, Autobuses nocturnos, Atascos, Pasos de peatones, Carriles bici, Medios de transporte, Aparcamientos, Peatones, Peatonalizacion, Lineas de metro, Taxis, Movilidad y medio ambiente, Bicimad, Transporte publico \\
		\hline \hline
		\hline
	\end{tabular}
\end{table}


\begin{table}[Relación de categorías y subcategorías para análisis semántico (III)]{TB:NAMEDENTITIESTOPICS3}{Tercera parte de la tabla donde se muestran las categorías y subcategorías declaradas para el análisis semántico de los tweets.}
	
	\begin{tabular}{|p{4.5cm}|p{12cm}|}
		\hline
		\textbf{Categoría (Named Entity)} & \textbf{Subcategorías (Topics)} \\
		Ocio y entretenimiento &  Ocio y tiempo libre, Ocio infantil, Ocio nocturno, Fiestas, Conciertos, Festivales, Eventos, Hosteleria, Terrazas, Cines, Teatros, Juegos, Zoo \\
		Participación ciudadana &  Sensibilizacion, Foros locales, Emponderamiento de la ciudadania, Atencion al ciudadano, Presupuestos participativos, Participacion ciudadana \\
		Política &  Referendum, Democracia, Elecciones, Politicos \\
		Religión &  Laicismo, La iglesia, Islam, Belenes, Cabalgata \\
		Salud y sanidad &  Aparcamiento en hospitales, Personal sanitario, Optometria, Dentistas, Enfermedades raras, Alergias, Donantes, Nutricion, Bebidas alcoholicas, Drogas, Ludopatia, Actividad fisica, Sanidad publica, Enfermedades, Hospitales \\
		Seguridad y emergencias &  Seguridad vial, Vigilancia, Emergencias, Peligrosidad, Bomberos, Guardia civil, Ejercito, Ambulancias, Policia, Seguridad \\
		Sostenibilidad &  Sostenibilidad, Reciclaje, Ahorro energetico, Energias renovables, Huertos urbanos, Car sharing, Diesel \\
		Tercera edad &  Intergeneracionalidad, Jubilación, Residencias, Mayores \\
		Transparencia &  Transparencia, Gobierno abierto, Datos abiertos \\
		Turismo &  Turismo, Turistificacion, Oferta turistica, Patrimonio, Viviendas turisticas, Autobuses turisticos, Tasa turistica \\
		Urbanismo &  Urbanismo y movilidad, Urbanismo y medio ambiente, Bienes publicos, Espacios publicos, Diseño urbano, Estetica urbana, Mobiliario urbano, Alumbrado, Edificios, Obras, Ordenamiento urbano, Urbanismo, Fachadas, Carreteras, Calzada, Baches, Caminos, Puentes, Fuentes, Baños publicos, Antenas, Descampados, Monumentos franquistas, Bolardos, Calles y aceras, Nombres de lugares, Rotondas, Vias publicas \\
		Vivienda &  Derecho a una vivienda, Vivienda publica, Cooperativas, Compra de vivienda, Plusvalia, Alquiler de vivienda, IBI, Inmuebles \\
		\hline \hline
		\hline
	\end{tabular}
\end{table}


\begin{table}[Resultados de la clasificación de los tweets semánticamente por categorías.]{TB:REPORTBYNAMEDENTITIES}{Se muestran las categorías a las que pertenecen los tweets}
	
	\begin{tabular}{|p{8cm}|c|}
		\hline
		\textbf{Categoría (Named Entity)} & \textbf{Cantidad de tweets} \\
		Participación ciudadana & 2540 \\
		Política & 1476 \\
		Ayuntamiento y administración pública & 1335 \\
		Movilidad & 906 \\
		Urbanismo & 467 \\
		Medio ambiente & 306 \\
		Seguridad y emergencias & 242 \\
		Derechos sociales & 219 \\
		Civismo & 213 \\
		Economía & 157 \\
		Tercera edad & 94 \\
		Salud y sanidad & 81 \\
		Vivienda & 77 \\
		Equidad e integración & 76 \\
		Asociaciones & 71 \\
		Empleo & 52 \\
		Familia e infancia & 41 \\
		Educación & 31 \\
		\hline \hline
		\hline
	\end{tabular}
\end{table}

\begin{table}[Resultados de la clasificación de los tweets semánticamente por subcategorías (I).]{TB:REPORTBYTOPICS1}{Se muestran las subcategorías a las que pertenecen los tweets.}
\begin{tabular}{|c|c|}
	\hline
	\textbf{Subcategoría (Topic)} & \textbf{Cantidad de tweets} \\
	participacion ciudadana & 2550 \\
	politicos & 1455 \\
	gestion publica & 811 \\
	medios de transporte & 548 \\
	barrios & 432 \\
	vias publicas & 231 \\
	convivencia & 213 \\
	elecciones & 205 \\
	presupuestos participativos & 193 \\
	transporte publico & 183 \\
	policia & 183 \\
	carriles bici & 179 \\
	movilidad y seguridad & 179 \\
	carril bus & 179 \\
	equipo de gobierno & 170 \\
	movilidad en bicicleta & 155 \\
	metro & 150 \\
	accesibilidad peatonal & 141 \\
	dinero publico & 141 \\
	democracia & 123 \\
	zonas verdes & 123 \\
	movilidad y medio ambiente & 118 \\
	calles y aceras & 116 \\
	medio ambiente y limpieza & 109 \\
	servicios de limpieza & 109 \\
	ordenanza municipal & 105 \\
	medio ambiente & 104 \\
	ayudas economicas & 101 \\
	ayudas sociales & 101 \\
	mayores & 97 \\
	\hline \hline
	\hline
\end{tabular}
\end{table}

\begin{table}[Resultados de la clasificación de los tweets semánticamente por subcategorías (II).]{TB:REPORTBYTOPICS2}{Segunda parte de la tabla donde se muestran las subcategorías a las que pertenecen los tweets.}
	\begin{tabular}{|c|c|}
		\hline
		\textbf{Subcategoría (Topic)} & \textbf{Cantidad de tweets} \\
		contenedores & 85 \\
	basuras & 85 \\
	pasos de peatones & 85 \\
	obras & 84 \\
	inmuebles & 78 \\
	vehiculos & 77 \\
	mujeres & 76 \\
	asociacionismo & 71 \\
	seguridad & 67 \\
	hospitales & 66 \\
	atascos & 66 \\
	derechos humanos & 64 \\
	servicios sociales & 61 \\
	empleo & 54 \\
	bolardos & 51 \\
	ordenamiento urbano & 51 \\
	empresas & 50 \\
	sanidad publica & 44 \\
	rotondas & 41 \\
	taxis & 41 \\
	familias & 41 \\
	peatones & 40 \\
	peatonalizacion & 40 \\
	lineas de metro & 37 \\
	enfermedades & 36 \\
	colegios & 31 \\
	derechos sociales y educacion & 29 \\
	parques y jardines & 20 \\
	urbanismo & 17 \\
	aparcamientos & 6 \\
		\hline \hline
		\hline
	\end{tabular}
\end{table}