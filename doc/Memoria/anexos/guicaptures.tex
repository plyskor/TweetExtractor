\begin{figure}[Medidas con SonarQube]{FIG:SONARMEASURES}{Algunas medidas ofrecidas por SonarQube}
	\subfigure[SBFIG:SONARMEASURES2]{Duplicaciones, tamaño y complejidad}{\image{4.2cm}{}{sonar/sonarMeasures2}} \quad
	\subfigure[SBFIG:SONARMEASURES1]{Confiabilidad y seguridad}{\image{4.2cm}{}{sonar/sonarMeassures1}} \quad
	\subfigure[SBFIG:SONARMEASURES3]{Mantenibilidad y cobertura de tests}{\image{4.2cm}{}{sonar/sonarMeassures3}}
\end{figure}

\begin{figure}[Pantalla inicial]{FIG:WELCOMESCREEN}{Pantalla principal de bienvenida al abrir la aplicación}
	\image{10cm}{}{gui/homeScreen}
\end{figure}
\begin{figure}[Pantalla de bienvenida]{FIG:HOMESCREEN}{Pantalla de inicio (Menú principal)}
	\image{10cm}{}{gui/welcomeScreen}
\end{figure}

\begin{figure}[Edición de credenciales]{FIG:EDITCREDENTIALS}{Configurando los credenciales de la API de Twitter (los credenciales mostrados en la foto no son reales)}
	\image{10cm}{}{gui/editCredentials}
\end{figure}

\begin{figure}[Editor de extracciones]{FIG:SHOWEXTRACTIONDETAILS}{Pantalla de visualización edición y exportación de extracciones.}
	\image{}{}{gui/showExtractionDetails}
\end{figure}


\begin{figure}[Tarea en segundo plano]{FIG:LOADINGDIALOG}{Las tareas costosas no bloquean la aplicación. Un nuevo hilo nos muestra un diálogo informativo sobre lo que se está haciendo.}
	\image{10cm}{}{gui/backgroundDBTask}
\end{figure}


\begin{figure}[Aviso sobre extracción ya en curso]{FIG:CONCURRENTEXTRACTIONDIALOG}{Diálogo que nos informa de que una extracción ya está siendo alimentada por otro proceso.}
	\image{8cm}{}{gui/concurrentExtractionDialog}
\end{figure}


\begin{figure}[Menú GUI de gestion de reportes analíticos]{FIG:MYANALYTICSREPORTS}{Menú desde el que se gestionan los reportes analíticos.}
	\image{}{}{gui/myAnalyticsReports}
\end{figure}

\begin{figure}[Menú GUI de gestion de las gráficas]{FIG:MYGRAPHICCHARTS}{Menú desde el que se gestionan los gráficos generados a partir de los reportes analíticos.}
	\image{}{}{gui/myGraphicCharts}
\end{figure}

\begin{figure}[Menú de selección de tipo de gráfico]{FIG:CHARTTYPES}{Menú que muestra los tipos de gráficos que se pueden generar a partir de los reportes.}
	\image{}{}{gui/availableChartTypes}
\end{figure}

\begin{figure}[Configuración de un gráfico (general)]{FIG:GENERICCHARTCONFIG}{Menú genérico de configuración de gráficos}
	\image{}{}{gui/chartConfig}
\end{figure}

\begin{figure}[Configuración de un gráfico (específico)]{FIG:SPECIFICCHARTCONFIG}{Menús de configuraciones específicas para distintos tipos de gráficos}
		\subfigure[SBFIG:XYCHARTCONFIG]{Gráfico XY}{\image{}{}{gui/XYChartConfig}} \quad
	\subfigure[SBFIG:SONARMEASURES1]{Gráfico de barras 3D}{\image{}{}{gui/categoryBarChartConfig}} \quad
\end{figure}

\begin{figure}[Configuración de trazos del gráfico por categoría]{FIG:PLOTSTROKECONFIURATION}{Menú donde se configura cómo se pintará cada categoría en el gráfico.}
	\image{}{}{gui/PlotStrokeConfig}
\end{figure}

\begin{figure}[Mismo reporte, distintos gráficos]{FIG:SAMEREPORTCHARTS}{Diferentes  gráficos generados sobre el mismo reporte que muestra la intercompatibilidad.}
\subfigure[SBFIG:TIMESERIESCHART]{Línea del tiempo (tweets/día)}{\image{15cm}{}{gui/TimeSeriesChart}} \quad
\subfigure[SBFIG:XYBARCHART]{Barras XY (tweets/día)}{\image{15cm}{}{gui/XYBarChart}} 
\end{figure}

\begin{figure}[Diferentes gráficos generados ]{FIG:GENERATEDCHARTS}{Se muestran distintos tipos de gráficos para distintos reportes de tendencia.}
	\subfigure[SBFIG:3DBARCHART]{Barras 3D (Tendencias en hashtags)}{\image{15cm}{}{gui/3DBarChart}} \quad
	\subfigure[SBFIG:3DPIECHART]{Tarta 3D (tendencias en menciones)}{\image{15cm}{}{gui/3DPieChart}} 
\end{figure}


\begin{figure}[Editando una configuración para el análisis semántico]{FIG:EDITNERCONFIGURATION}{Menú de la GUI desde el cual se pueden configurar las categorías y subcategorías para el análisis semántico.}
	\image{}{}{gui/NERConfigurationEdit}
\end{figure}

